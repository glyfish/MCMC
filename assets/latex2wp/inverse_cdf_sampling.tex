\documentclass[12pt]{article}
\usepackage[pdftex,pagebackref,colorlinks=true,pdfpagemode=none,urlcolor=blue,linkcolor=blue,citecolor=blue,pdfstartview=FitH]{hyperref}

\usepackage{amsmath,amsfonts}
\usepackage{graphicx}
\usepackage{color}
\usepackage{hyperref}
\usepackage{minted}

\setlength{\oddsidemargin}{0pt}
\setlength{\evensidemargin}{0pt}
\setlength{\textwidth}{6.0in}
\setlength{\topmargin}{0in}
\setlength{\textheight}{8.5in}

\setlength{\parindent}{0in}
\setlength{\parskip}{5px}

%%%%%%%%% For wordpress conversion

\def\more{}

\newif\ifblog
\newif\iftex
\blogfalse
\textrue


\usepackage{ulem}
\def\em{\it}
\def\emph#1{\textit{#1}}

\def\image#1#2#3{\begin{center}\includegraphics[#1pt]{#3}\end{center}}

\let\hrefnosnap=\href

\newenvironment{btabular}[1]{\begin{tabular} {#1}}{\end{tabular}}

\newenvironment{red}{\color{red}}{}
\newenvironment{green}{\color{green}}{}
\newenvironment{blue}{\color{blue}}{}

%%%%%%%%% Typesetting shortcuts

\def\B{\{0,1\}}
\def\xor{\oplus}

\def\P{{\mathbb P}}
\def\E{{\mathbb E}}
\def\var{{\bf Var}}

\def\N{{\mathbb N}}
\def\Z{{\mathbb Z}}
\def\R{{\mathbb R}}
\def\C{{\mathbb C}}
\def\Q{{\mathbb Q}}
\def\eps{{\epsilon}}

\def\bz{{\bf z}}

\def\true{{\tt true}}
\def\false{{\tt false}}

%%%%%%%%% Theorems and proofs

\newtheorem{exercise}{Exercise}
\newtheorem{theorem}{Theorem}
\newtheorem{lemma}[theorem]{Lemma}
\newtheorem{definition}[theorem]{Definition}
\newtheorem{corollary}[theorem]{Corollary}
\newtheorem{proposition}[theorem]{Proposition}
\newtheorem{example}{Example}
\newtheorem{remark}[theorem]{Remark}
\newenvironment{proof}{\noindent {\sc Proof:}}{$\Box$ \medskip} 


\title{Inverse CDF Sampling}
\author{Troy Stribling}

\begin{document}

\iftex
\maketitle
\fi

\section{Introduction}

Inverse \href{https://en.wikipedia.org/wiki/Cumulative_distribution_function}{CDF} sampling is a method for obtaining samples from both discrete and continuous probability distributions
that requires the CDF to be invertable.
The method proposes a CDF value from a Uniform random variable on $\ [0, 1]\ $ which is then used as input
into the inverted CDF to generate a sample
with the desired discrete or continuous distribution. Here examples for both cases are discussed.
For the continuous case a proof is given that demonstrates the samples produced have the expected distribution.

\section{Sampling Discrete Distributions}

A discrete probability distribution consisting of a finite set of $N$ probability values is defined by,

\begin{equation}
\label{eq:discrete_distribution}
\{ p_1, p_2,\ldots,p_N\}
\end{equation}

with,

$$\sum_{i=1}^N{p_i} = 1.$$

The CDF specifies the probability that $i \leq n$ and is given by,
\begin{equation}
\label{eq:discrete_cdf}
P(n)=\sum_{i=1}^n{p_i},
\end{equation}
where $P(N)=1.$

For a given CDF propsal, $P^*$, equation (\ref{eq:discrete_cdf}) can always be inverted by evaluating it for each $n$ and
searching for the value of $n$ that satisfies,
$$P(n) \geq P^*.$$

A sampler can be implemented in Python with the following,

\ifblog
<pre class="EnlighterJSRAW" data-enlighter-language="python" data-enlighter-linenumbers="false">
import numpy
nsamples = 100000
cdf_proposals = numpy.random.rand(nsamples)
samples = [numpy.flatnonzero(cdf >= cdf_proposals[i])[0] for i in range(nsamples)]
</pre>
\fi

\iftex
\begin{minted}[mathescape, frame=lines, framesep=2mm, fontsize=\footnotesize]{python}
import numpy

nsamples = 100000
cdf_proposals = numpy.random.rand(nsamples)
samples = [numpy.flatnonzero(cdf >= cdf_proposals[i])[0] for i in range(nsamples)]
\end{minted}
\fi

Consider the following disrete distribution,

\begin{equation} \label{eq:discrete}
\left \{\frac{1}{12}, \frac{1}{12}, \frac{1}{6}, \frac{1}{6}, \frac{1}{12}, \frac{5}{12} \right\}
\end{equation}



\end{document}
