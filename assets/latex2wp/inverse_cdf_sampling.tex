\documentclass[12pt]{article}
\usepackage[pdftex,pagebackref,colorlinks=true,pdfpagemode=none,urlcolor=blue,linkcolor=blue,citecolor=blue,pdfstartview=FitH]{hyperref}

\usepackage{amsmath,amsfonts}
\usepackage{graphicx}
\usepackage{color}


\setlength{\oddsidemargin}{0pt}
\setlength{\evensidemargin}{0pt}
\setlength{\textwidth}{6.0in}
\setlength{\topmargin}{0in}
\setlength{\textheight}{8.5in}

\setlength{\parindent}{0in}
\setlength{\parskip}{5px}

\begin{document}

Inverse CDF sampling is a method for obtaining samples from both discrete and continuous probability distributions that requires the CDF to be invertable.
The method proposes a CDF value from a Uniform random variable on [0, 1] which is then used as input into the inverted CDF to generate a sample
with the desired discrete or continuous distribution. Here examples for both cases are discussed.
For the continuous case a proof is given that demonstrates the samples produced have the expected distribution.

Sampling Discrete Distributions

A discrete probability distribution consisting of a finite set of $N$ probability values is defined by,

$$p=\{ p_1, p_2,\ldots,p_N\}$$

with,

$$\sum_{i=1}^N{p_i} = 1$$

The CDF specifies the probability that $i \leq n$ and is given by,

\begin{equation} \label{eq:discrete_cdf}
P(n)=\sum_{i=1}^n{p_i},
\end{equation}

where $P(N)=1$.

Consider the following disrete distribution,

\begin{equation} \label{eq:discrete}
p =\left \{\frac{1}{12}, \frac{1}{12}, \frac{1}{6}, \frac{1}{6}, \frac{1}{12}, \frac{5}{12} \right\}
\end{equation}

\end{document}
