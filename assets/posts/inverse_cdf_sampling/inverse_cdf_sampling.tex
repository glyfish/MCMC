\documentclass[12pt]{article}
\usepackage[pdftex,pagebackref,colorlinks=true,pdfpagemode=none,urlcolor=blue,linkcolor=blue,citecolor=blue,pdfstartview=FitH]{hyperref}

\usepackage{amsmath,amsfonts}
\usepackage{graphicx}
\usepackage{color}
\usepackage{hyperref}
\usepackage{minted}
\usemintedstyle{bw}

\newcommand{\homedir}{\string~}

\setlength{\oddsidemargin}{0pt}
\setlength{\evensidemargin}{0pt}
\setlength{\textwidth}{6.0in}
\setlength{\topmargin}{0in}
\setlength{\textheight}{8.5in}

\setlength{\parindent}{0in}
\setlength{\parskip}{5px}

%%%%%%%%% For wordpress conversion

\def\more{}

\newif\ifblog
\newif\iftex
\blogfalse
\textrue


\usepackage{ulem}
\def\em{\it}
\def\emph#1{\textit{#1}}

\def\image#1#2#3{\begin{center}\includegraphics[#1pt]{#3}\end{center}}

\let\hrefnosnap=\href

\newenvironment{btabular}[1]{\begin{tabular} {#1}}{\end{tabular}}

\newenvironment{red}{\color{red}}{}
\newenvironment{green}{\color{green}}{}
\newenvironment{blue}{\color{blue}}{}

%%%%%%%%% Typesetting shortcuts

\def\B{\{0,1\}}
\def\xor{\oplus}

\def\P{{\mathbb P}}
\def\E{{\mathbb E}}
\def\var{{\bf Var}}

\def\N{{\mathbb N}}
\def\Z{{\mathbb Z}}
\def\R{{\mathbb R}}
\def\C{{\mathbb C}}
\def\Q{{\mathbb Q}}
\def\eps{{\epsilon}}

\def\bz{{\bf z}}

\def\true{{\tt true}}
\def\false{{\tt false}}

%%%%%%%%% Theorems and proofs

\newtheorem{exercise}{Exercise}
\newtheorem{theorem}{Theorem}
\newtheorem{lemma}[theorem]{Lemma}
\newtheorem{definition}[theorem]{Definition}
\newtheorem{corollary}[theorem]{Corollary}
\newtheorem{proposition}[theorem]{Proposition}
\newtheorem{example}{Example}
\newtheorem{remark}[theorem]{Remark}
\newenvironment{proof}{\noindent {\sc Proof:}}{$\Box$ \medskip} 


\title{Inverse CDF Sampling}
\author{Troy Stribling}


\begin{document}

% Introduction
\iftex
\maketitle
\section{Introduction}
\fi

Inverse \href{https://en.wikipedia.org/wiki/Cumulative_distribution_function}{CDF} sampling is a method for obtaining samples
from both discrete and continuous probability distributions
that requires the CDF to be invertable. The method assumes values of the CDF are Uniform random variables on [0, 1].
Values are generated and used as input into the inverted CDF to obain samples with the distribution defined by the CDF.
Here examples for both continuous and discrete cases are discussed.

% Sampling discrete distributions
\ifblog
<h2>Sampling Discrete Distributions</h2>
\fi
\iftex
\section{Sampling Discrete Distributions}
\fi

A discrete probability distribution consisting of a finite set of $N$ probability values is defined by,
$\{p_1, p_2,\ldots,p_N\}$ with $p_i \geq 0, \forall i$ and $\sum_{i=1}^N{p_i} = 1.$ The CDF specifies the probability
that $i \leq n$ and is given by,
\begin{equation}
\label{eq:discrete_cdf}
P(i \leq n)=P(n)=\sum_{i=1}^n{p_i},
\end{equation}
where $P(N)=1.$

For a given generated CDF value, $u$, Equation (\ref{eq:discrete_cdf}) can always be inverted by evaluating it for each $n$ and
searching for the value of $n$ that satisfies, $P(n) \geq u.$ It can be seen that the generated samples will have
distribution $\{p_n\}$ since the intervals $P(n)-P(n-1) = p_n$ are Uniformly sampled.

Consider the distribution,

\begin{equation} 
\left \{\frac{1}{12}, \frac{1}{12}, \frac{1}{6}, \frac{1}{6}, \frac{1}{12}, \frac{5}{12} \right\}
\label{eq:discrete}
\end{equation}

It is shown in the following plot with its CDF.

\ifblog
\image{width = 600}{https://gly.fish/wp-content/uploads/posts/inverse-cdf-sampling/discrete_cdf.png}{\homedir/Develop/python/gly.fish/assets/posts/inverse_cdf_sampling/discrete_cdf.png}
\fi
\iftex
\image{width = 400}{https://gly.fish/wp-content/uploads/posts/inverse-cdf-sampling/discrete_cdf.png}{\homedir/Develop/python/gly.fish/assets/posts/inverse_cdf_sampling/discrete_cdf.png}
\fi

A sampler using the Inverse CDF method can be implemented in Python in a few lines of code,

% Sampler code examples
\ifblog
<pre class="EnlighterJSRAW" data-enlighter-language="python">

import numpy

n = 10000
df = numpy.array([1/12, 1/12, 1/6, 1/6, 1/12, 5/12])
cdf = numpy.cumsum(df)

cdf_star = numpy.random.rand(n)
samples = [numpy.flatnonzero(cdf >= cdf_star[i])[0] for i in range(n)]

</pre>
\fi

\iftex
\begin{minted}[mathescape, frame=lines, framesep=2mm, fontsize=\footnotesize]{python}
import numpy

n = 10000
df = numpy.array([1/12, 1/12, 1/6, 1/6, 1/12, 5/12])
cdf = numpy.cumsum(df)

cdf_star = numpy.random.rand(n)
samples = [numpy.flatnonzero(cdf >= cdf_star[i])[0] for i in range(n)]
\end{minted}
\fi

The figure below favorably compares generated samples and distribution (\ref{eq:discrete}),
\ifblog
\image{width = 600}{https://gly.fish/wp-content/uploads/posts/inverse-cdf-sampling/discrete_sampled_distribution.png}{\homedir/Develop/python/gly.fish/assets/posts/inverse_cdf_sampling/discrete_sampled_distribution.png}
\fi
\iftex
\image{width = 400}{https://gly.fish/wp-content/uploads/posts/inverse-cdf-sampling/discrete_sampled_distribution.png}{\homedir/Develop/python/gly.fish/assets/posts/inverse_cdf_sampling/discrete_sampled_distribution.png}
\fi


% multinomial sampling
\ifblog
It is also possible to directly sample $\{p_n\}$ using the <code>multinomial</code> sampler from
<code>numpy</code>,

<pre class="EnlighterJSRAW" data-enlighter-language="python">
import numpy

n = 10000
df = numpy.array([1/12, 1/12, 1/6, 1/6, 1/12, 5/12])
samples = numpy.random.multinomial(n, df, size=1)/n
</pre>
\fi

\iftex
It is also possible to directly sample $\{p_n\}$ using the \mintinline{python}{multinomial} sampler from
\mintinline{python}{numpy},

\begin{minted}[mathescape, frame=lines, framesep=2mm,fontsize=\footnotesize]{python}
import numpy

n = 10000
df = numpy.array([1/12, 1/12, 1/6, 1/6, 1/12, 5/12])
samples = numpy.random.multinomial(n, df, size=1)/n
\end{minted}
\fi

% Sampling continuous distributions
\ifblog
<h2>Sampling Continuous Distributions</h2>
\fi
\iftex
\section{Sampling Continuous Distributions}
\fi

A continuous probability distribution is defined by the  \href{https://en.wikipedia.org/wiki/Probability_density_function}{PDF},
$f_X(x)$, where $f_X(x) \geq 0, \forall x$ and $\int f_X(x) dx = 1.$ The CDF is a monotonically increasing function
that specifies probability that $X \leq x$,
\begin{equation}
\label{eq:continuous_cdf}
P(X \leq x) = F_X(x) = \int^{x} f_X(w) dw
\end{equation}

To prove that Inverse CDF sampling works it must be shown that,

\begin{equation}
\label{eq:continuous_proof}
P[F_X^{-1}(u) \leq x] = F_X(x),
\end{equation}

where $F_X^{-1}(x)$ is the inverse of $F_X(x)$ and $u$ is $\textbf{Uniform}(0, 1)$. A result needed to complete this proof
is the following resulting from a change of variables. If $Y=G(X)$ is a monotonically increasing invertable function of $x$ then
\begin{equation}
\label{eq:CDF_invariance}
P(X \leq x) = P[G(X) \leq G(x)] = P(Y \leq y).
\end{equation}

$G(x)$ is monotonically increasing so the ordering of values is preserved,

$$ X \le x & \implies G(X) \le G(x).$$

Consequently the order of the integration limits is maintained. Because $G(x)$ is invertable,
$x = G^{-1}(y)$ and $dx = \frac{dG^{-1}}{dy} dy$, so

$$
\begin{aligned}
P(X \leq x) & = \int^{x} f_X(w) dw \\
& = \int^{y} f_X(G^{-1}(z)) \frac{dG^{-1}}{dz} dz \\
& = P(Y \leq y) \\
& = P[G(X) \leq G(x)],
\end{aligned}
$$

where it is assumed that,

$$
f_Y(y) = f_X(G^{-1}(y)) \frac{dG^{-1}}{dy}
$$

The proof of Equation (\ref{eq:continuous_proof}) follows from Equation (\ref{eq:CDF_invariance}), using $f_U(u) = 1$

$$
\begin{aligned}
P[F_X^{-1}(u) \leq x] & = P[F_X(F_X^{-1}(u)) \leq F_X(x)] \\
& = P[u \leq F_X(x)] \\
& = \int_{0}^{F_X(x)} f_U(w) dw \\
& = \int_{0}^{F_X(x)} dw \\
& = F_X(x)
\end{aligned}
$$

As an example consider the \href{https://en.wikipedia.org/wiki/Weibull_distribution}{Weibull Distribution},

\begin{equation}
\label{eq:Weibull_distribution}
f_x(x)
\end{equation}

\end{document}
